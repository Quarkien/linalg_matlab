\documentclass[a4paper]{article}
\usepackage{amsmath}
\usepackage{amsthm}
\newtheorem{thm}{Sats}
\newtheorem{dfn}{Definition}
\usepackage[english]{babel}
\usepackage[utf8]{inputenc}
\usepackage{graphicx}
\usepackage[colorinlistoftodos]{todonotes}

\title{Analys i flera variabler}

\author{Alexandr Djadkin}

\date{\today}
\begin{document}
\maketitle




\section{Kap 6 Dubbelintegraler}

\begin{dfn}

Den begränsade funktionen $f(x,y)$ är \textbf{(Riemann)integrerbar} över rektangeln $\Delta$ om det till varje tal $\epsilon > 0$ finns trappfunktioner $\Phi$ och $\Psi$ sådana att $\Phi \leq f \leq \Psi$ och $\iint_\Delta \Phi \,dx \,dy - \iint_\Delta \Psi \,dx \,dy < \epsilon$

\end{dfn}


\begin{thm}
Om $f$ är integrerbar över $\Delta$ så finns precis ett tal $\lambda$ med egenskapen att  $\iint_\Delta \Phi\,dx\,dy \leq \lambda \leq \iint_\Delta \Psi\,dx\,dy$  för alla trappfunktioner $\Phi$ och $\Psi$ med $\Phi \leq f \leq \Psi$.

\end{thm}


\subsection{Specialfall}

\begin{equation}
\iint_\Delta g(x)h(x )\,dx \,dy = \left(\int_{a}^{b} g(x) dx\right)\left(\int_{a}^{b} h(y) dy\right)
\end{equation} 

\subsection{Monotonitetsegenskapen}


$\Phi \leq \Psi $ på $ \Delta \Rightarrow \iint_\Delta \Phi \,dx \,dy \leq \iint_\Delta \Psi \,dx \,dy\
$

\subsection{6.4 Variabelbyte i dubbelintegraler}

\textbf{Formeln för variabelsubstitution i enkelintegral}

Om $ x = g(t) $ är $ \int_{a}^{b} f(x) dx = \pm \int_{\alpha}^{\beta} f(g(t))g'(t)dt $
, där tecknet beror på om g är växande eller avtagande.


\end{document}
