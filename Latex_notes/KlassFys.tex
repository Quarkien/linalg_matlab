\documentclass[a4paper]{article}
\usepackage{amsmath}
\usepackage{amsthm}
\newtheorem{thm}{Sats}
\newtheorem{dfn}{Definition}
\usepackage[english]{babel}
\usepackage[utf8]{inputenc}
\usepackage{graphicx}
\usepackage[colorinlistoftodos]{todonotes}

\title{Klassisk Fysik}

\author{Alexandr Djadkin}

\date{\today}
\begin{document}
\maketitle




\section{Kap 23.5 Potential gradienten}

Om det elektriska fältet $\vec{E}$ är känt som en funktion av positionen kan potentialen V räknas ut med  \\[2mm]

$V_a - V_b = \int_{a}^{b} \vec{E} d\vec{l} = \int_{a}^{b} Ecos\phi dl$. 
\\[2mm]
Om V är känd som funktion av positionen kan $\vec{E}$ räknas ut med 
\\[2mm]

$ E_x = -\dfrac{\partial V}{\partial x}, \qquad E_x = -\dfrac{\partial V}{\partial y}, \qquad E_x = -\dfrac{\partial V}{\partial z}$
\\[2mm]
Radiellt elektriskt fält med avseende på en punkt eller axel, med r som avstånd:

$ E_r = -\dfrac{\partial V}{\partial r}$

\section{Kap 24.4 Dielektrika}

Kapacitans: $ C = \dfrac{Q}{V}=\epsilon_0\dfrac{A}{d}$ , där $Q$ är laddning, $V$ är potential, $A$ är plattornas ytarea och $d$ är avstånden mellan plattorna. $\epsilon_0 = 8,85*10^{-12} F/m$
\\[2mm]
Dielektriskt konstant: $K = \dfrac{C}{C_0}$, med $C_0$ som kapacitansen i vakuum. Potentialen och det elektriska fältet minskar med en faktor $K$, samtidigt som kapacitansen ökar med en faktor $K$
\\[2mm]
Potential: $V = \dfrac{V_0}{K}$, med $V_0$ som potentialen i vakuum.
\\[2mm]
Elektriskt fält mellan plattorna: $E = \dfrac{E_0}{K}$ (när $Q$ är konstant)
\\[1mm]
Fältet mellan plattorna beror på netto ytladdningsdensiteten: 
$E=\sigma_{net}/\epsilon_0$
\\[1mm]
Med dielektrika: $E = \dfrac{\sigma - \sigma_i}{\epsilon_0}$, där $\sigma_i$ är på dielektrikan inducerad laddning per areaenhet.
\newpage

\section{Kap 25 Ström, Resistans och Elektromotorisk Kraft}

\subsection{Kap 25.1 Ström}

Definitionen av ström: $I = \dfrac{dQ}{dt}$ (genom en tvärsnittsarea)
\\[4mm]

\textbf{Drifthastighet och strömtäthet}

Ström genom en area: $I = \dfrac{dQ}{dt} = n\vert q \vert v_dA$, där $n$ är koncentrationen av laddade partiklar i rörelse, $q$ är laddning per partikel, $v_d$ är drifthastigheten och $A$ är tvärsnittsarean.
\\[2mm]

Strömtäthet: $J = \dfrac{I}{A} = n\vert q \vert v_d$, som också kan skrivas i vektorform där drifthastigheten ger riktningen (inget absolutbelopp i vektorekvationen). Om laddningen är positiv är drifthastigheten riktad åt samma håll som det elektriska fältet. Omvänt för negativ laddning. $\vec{J}$ är riktad i det elektriska fältets riktning.

Strömtäthet som vektor: $\vec{J} =  nq\vec{v_d}$









\end{document}