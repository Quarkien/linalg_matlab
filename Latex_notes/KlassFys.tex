
\documentclass[a4paper]{article}
\usepackage{amsmath}
\usepackage{amsthm}
\newtheorem{thm}{Sats}
\newtheorem{dfn}{Definition}
\usepackage[english]{babel}
\usepackage[utf8]{inputenc}
\usepackage{graphicx}
\usepackage[colorinlistoftodos]{todonotes}

\title{Klassisk Fysik}

\author{Alexandr Djadkin}
\begin{document}
\maketitle




\section*{23.5 Potential gradienten}

Om det elektriska fältet $\vec{E}$ är känt som en funktion av positionen kan potentialen V räknas ut med  
\\[2mm]
\begin{equation}
V_a - V_b = \int_{a}^{b} \vec{E} d\vec{l} = \int_{a}^{b} Ecos\phi dl
\end{equation}
\\[2mm]
Om V är känd som funktion av positionen kan $\vec{E}$ räknas ut med 
\\[2mm]
\begin{equation}
E_x = -\dfrac{\partial V}{\partial x}, \qquad E_x = -\dfrac{\partial V}{\partial y}, \qquad E_x = -\dfrac{\partial V}{\partial} z
\end{equation}
\\[2mm]
Radiellt elektriskt fält med avseende på en punkt eller axel, med r som avstånd:
\begin{equation}
E_r = -\dfrac{\partial V}{\partial r}
\end{equation}


\section*{24.4 Dielektrika}

Kapacitans: \begin{equation}
C = \dfrac{Q}{V}=\epsilon_0\dfrac{A}{d},
\end{equation} där $Q$ är laddning, $V$ är potential, $A$ är plattornas ytarea och $d$ är avstånden mellan plattorna. $\epsilon_0 = 8,85*10^{-12} F/m$
\\[2mm]
Dielektriskt konstant: $K = \dfrac{C}{C_0}$, med $C_0$ som kapacitansen i vakuum. Potentialen och det elektriska fältet minskar med en faktor $K$, samtidigt som kapacitansen ökar med en faktor $K$
\\[2mm]
Potential: $V = \dfrac{V_0}{K}$, med $V_0$ som potentialen i vakuum.
\\[2mm]
Elektriskt fält mellan plattorna: $E = \dfrac{E_0}{K}$ (när $Q$ är konstant)
\\[1mm]
Fältet mellan plattorna beror på netto ytladdningsdensiteten: 
$E=\sigma_{net}/\epsilon_0$
\\[1mm]
Med dielektrika: $E = \dfrac{\sigma - \sigma_i}{\epsilon_0}$, där $\sigma_i$ är på dielektrikan inducerad laddning per areaenhet.
\newpage

\section*{25 Ström, Resistans och Elektromotorisk Kraft}

\subsection*{25.1 Ström}

Definitionen av ström: $I = \dfrac{dQ}{dt}$ (genom en tvärsnittsarea)
\\[4mm]

\textbf{Drifthastighet och strömtäthet}

Ström genom en area: $I = \dfrac{dQ}{dt} = n\vert q \vert v_dA$, där $n$ är koncentrationen av laddade partiklar i rörelse, $q$ är laddning per partikel, $v_d$ är drifthastigheten och $A$ är tvärsnittsarean.
\\[2mm]

Strömtäthet: $J = \dfrac{I}{A} = n\vert q \vert v_d$, som också kan skrivas i vektorform där drifthastigheten ger riktningen (inget absolutbelopp i vektorekvationen). Om laddningen är positiv är drifthastigheten riktad åt samma håll som det elektriska fältet. Omvänt för negativ laddning. $\vec{J}$ är riktad i det elektriska fältets riktning.
\\
Strömtäthet som vektor: $\vec{J} =  nq\vec{v_d}$

\subsection*{25.2 Resistivitet}

\begin{equation}
\rho = \dfrac{E}{J}, 
\end{equation}
där E och J är beloppen av det elektriska fältet respektive strömtätheten.

\subsection*{Resistivitet och Temperatur}
\begin{equation}
\rho(T) = \rho_0[1 + \alpha(T - T_0)],
\end{equation}
där $\rho_0$ är resistiviteten vid en referenstemperatur $T_0$.

\subsection*{25.3 Resistans}

En ledares resistans:

\begin{equation}
R = \dfrac{\rho L}{A},
\end{equation}
där L är ledarens längd och A är tvärsnittsarea.
\\

\textbf{Ohms Lag}

\begin{equation}
V = IR
\end{equation}

\textbf{Resistans och temperatur}

\begin{equation}
R(T) = R_0[1 + \alpha(T - T_0)]
\end{equation}
\newpage


\subsection*{Elektromotorisk kraft och kretsar}
\begin{equation}
V_{ab} =  \mathcal{E} \quad \text{(ideal emf-källa)}
\end{equation}


\begin{equation}
\mathcal{E} = V_{ab} = IR \quad \text{(ideal emf-källa)}
\end{equation}

\subsection*{Inre Resistans}
\begin{equation}
V_{ab} = \mathcal{E} - Ir, \quad \text{där r är källans inre resistans.}
\end{equation}

\subsection*{25.5 Energi och effekt i elektriska kretsar}

\textbf{Effekt till eller från en komponent i kretsen}
\begin{equation}
P=V_{ab}I
\end{equation}

\textbf{Effekt levererad till en resistor}
\begin{equation}
P = V_{ab}I = I^2R = \dfrac{V_{ab}^2}{R}
\end{equation}

\textbf{Output av effekt}
\begin{equation}
P = V_{ab}I = \mathcal{E}I - I^2r
\end{equation}

Differansen $\mathcal{E}I - I^2r$ är netto ouput av elektrisk kraft från en källa.

\subsection*{25.6 Teorin om metallisk ledning}
\textbf{Resistivitet hos metall}
\begin{equation}
\rho = \dfrac{m}{ne^2\tau}, 
\end{equation}
där n är antalet fria elektroner per volym, e elektronladdningen, m elektronmassan och $\tau$ den genomsnittliga tiden mellan kollisioner.

\section*{26 likström och kretsar}

\subsection*{26.1 Resistorer i serie och parallelt}
\textbf{Resistorer i serie}
\begin{equation}
R_{eq} = R_1 + R_2 + R_3 + ...
\end{equation}
\textbf{Parallella resistorer}
\begin{equation}
\dfrac{1}{R_{eq}} = \dfrac{1}{R_{1}} + \dfrac{1}{R_{2}} + \dfrac{1}{R_{3}} + ... 
\end{equation}
\textbf{Specialfallet med två parallella resistorer}
\begin{equation}
R_{eq} = \frac{R_1R_2}{R_1+R_2}
\end{equation}

\begin{equation}
\dfrac{I_1}{I_2} = \dfrac{R_2}{R_1}
\end{equation}

\subsection*{26.2 Kirchhoff's lagar}
\textbf{Kirchhoff's regel om knutpunkt}
\begin{equation}
\sum V= 0
\end{equation}
Summan av alla strömmar in till en knutpunkt är 0.
\newline

\textbf{Kirchhoff's regel om loop}
\begin{equation}
\sum I = 0
\end{equation}
Summan av alla potentialskillnader i någon loop är 0.

\subsection*{26.3 Elektriska mätinstrument}
\textbf{Amperemeter}

En amperemeter kan anpassas till at mäta strömmar som är större än dess fullskaliga avläsning genom att koppla en resistor
(shuntmotstånd, $R_{sh}$)
parallellt med den så att en del av strömmen kringgår mätarspolen.

Antag att vi vill göra en mätare med fullskalig ström $I_{fs}$, vars spoles resistans är $R_c$ till en amperemätare med fullskalig avläsning $I_a$. $I_a$ är total ström genom parallellkopplingen, strömmen genom mätarens spole är $I_{fs}$ och strömmen genom shunten är differensen $I_a - I_{fs}$. Potentiallskillnaden $V_{ab}$ är samma för båda vägarna, så
\begin{equation}
I_{fs}R_c = (I_a - I_{fs})R_{sh} \quad \text{(för en amperemätare)}
\end{equation}

\textbf{Voltmätare}

För en voltmätare med fullskalig avläsning $V_V$ behövs en resistor i serie $R_s$ så att
\begin{equation}
V_V = I_{fs}(R_c + R_s) \quad \text{(för en voltmätare)}
\end{equation}

\subsection*{26.4 R-C Kretsar}

\textbf{Laddning av kondensator}
\begin{equation}
q = C \mathcal{E}(1-e^{-t/RC}) = Q_f(1-e^{-t/RC})
\end{equation}
q är kondensatorns laddning, C kapacitansen, $\mathcal{E}$ batteriets EMK, t tiden sedan brytaren stängts, R resistansen $Q_f$ är den slutliga laddningen hos kondensatorn (= $C\mathcal{E}$)
\\

Momentanströmmen är tidsderivatan av (25):
\begin{equation}
i = \dfrac{dq}{dt} = \dfrac{\mathcal{E}}{R} e^{-t/RC} = I_0 e^{-t/RC}
\end{equation}

$\dfrac{dq}{dt}$ är laddningens förändringshastighet och $I_0$ är initialströmmen. ($=\mathcal{E}/R$)

\newpage
\textbf{Tidskonstant}

Tidskonstanten är ett mått på hur snabbt en kondensator laddas. För litet värde på $\tau$ är laddningen snabb, för större värden tar laddningen längre tid.
\begin{equation}
\tau = RC
\end{equation}
\\
\textbf{Urladdning av kondensator}
\begin{equation}
q = Q_0e^{-t/RC}
\end{equation}

Momentanströmmen:
\begin{equation}
i = \dfrac{dq}{dt} = - \dfrac{Q_0}{RC} e^{-t/RC} = I_0 e^{-t/RC}
\end{equation}

\section*{27 Magnetiskt fält och magnetisk kraft}
\subsection*{27.2 Magnetiskt fält}

\textbf{Magetisk kraft på en laddad partikel i rörelse}
\begin{equation}
\vec{F} = q \vec{v} \times \vec{B} \quad \text{(Riktning ges av högerhandsregeln)}
\end{equation}

Kraftens storlek (belopp):

\begin{equation}
F = |q|vB_{\bot} = |q|vBsin\Phi \quad \text{($\Phi$ är vinkeln mellan $\vec{v}$ och $\vec{B}$)}
\end{equation}
Enhet för B: 1 tesla = 1 T = 1 $N/Am$. \\ 1 gauss = 1 G = $10^{-4}$ T.

\subsection*{27.3 Magnetiska fältlinjer och magnetiskt flöde}
\textbf{Magnetiskt flöde genom yta}
\begin{equation}
\Phi_B = \int B cos \Phi dA = \int B_\bot dA = \int \vec{B} * d \vec{A},
\end{equation}
där B är beloppet av det magnetiska fältet $\vec{B}$ och  $\Phi$ vinkeln mellan $\vec{B}$ och normalen till ytan.  \\[2 mm]
Enhet för $\Phi_B$: 1 weber = 1 wb = 1 $T*m^2$ = 1 $Nm/A$ 
\\

\textbf{Gauss's lag för magnetism}

Det totala magnetiska flödet genom någon sluten yta 
\begin{equation}
\oint \vec{B}*d \vec{A} = 0 \qquad \text{... är lika med 0.}
\end{equation}

\subsection*{27.4 Laddade partiklars rörelse i magnetiska fält}
\textbf{Radien för cirkulär bana i magnetiskt fält}
\begin{equation}
R = \dfrac{mv}{|q|B}
\end{equation}

\textbf{Vinkelhastighet}
\begin{equation}
\omega = \dfrac{v}{R} = v \dfrac{|q|B}{mv} = \dfrac{|q|B}{m}
\end{equation}

\subsection*{27.5 Tillämpningar av laddade partiklars rörelse}
...

\subsection*{27.6 Magnetisk kraft på en strömförande ledare}
\textbf{Magnetisk kraft på ett rakt trådsegment}
\begin{equation}
\vec{F} = I \vec{l} \times \vec{B}
\end{equation}

\textbf{Magnetisk kraft på ett infinitesimalt trådsegment}
\begin{equation}
d \vec{F} = I d \vec{l} \times \vec{B}
\end{equation}

\subsection*{27.7 Kraft och vridmoment på en strömloop}

\textbf{Magnetiskt vridmoment på strömloop}
\begin{equation}
\tau = IBAsin\phi,
\end{equation}
där $I$ är strömmen, $B$ magnetiska fältstyrkan, $A$ är arean för loopen och $\phi$ är vinkeln mellan normalen till loopplanet och fältriktiningen.

Produkten $IA$ kallas för \textbf{magnetiskt moment} för loopen. En strömloop eller annan kropp som påverkas av magnetiskt vridmoment kallas för \textbf{magnetisk dipol}.







\end{document}